\documentclass[letterpaper,14pt,twoside=true]{scrbook}

\usepackage[math]{anymouse}

\makeatletter
\newcommand*{\rom}[1]{\expandafter\@slowromancap\romannumeral #1@}
\makeatother

\DeclareMathSymbol{\Omega}{\mathalpha}{letters}{"0A}% italics
\DeclareMathSymbol{\varOmega}{\mathalpha}{operators}{"0A}% upright
\providecommand*{\upOmega}{\varOmega}% for siunitx Ohm
\usepackage{siunitx}
\sisetup{per-mode=symbol,per-symbol = p}
\DeclareSIUnit\bit{b}
\DeclareSIUnit\awg{AWG}
\DeclareSIUnit\mile{mi}
\DeclareSIUnit\vrms{\ensuremath{\mathrm{V_{rms}}}}

\usepackage{graphicx}
%\graphicspath{{figures/}}

\newcommand{\textdef}[1]{\textbf{\textit{#1}}}
\newcommand{\textmdef}[1]{\textit{#1}}

\newcommand{\myhref}[2]{\href{#1}{#2}\footnote{\url{#1}}}
\newcommand{\myamazon}[2]{\href{https://www.amazon.com/dp/#1}{#2}\footnote{\url{https://www.amazon.com/dp/#1}}}
% lazy way to make sure non-breaking spaces are used for versions
\newcommand{\atsc}[1]{ATSC~#1}
\newcommand{\moca}[1]{MoCA~#1}
\newcommand{\wifi}[1]{WiFi~#1}
\newcommand{\ieee}[1]{IEEE~#1}
\newcommand{\docsis}[1]{DOCSIS~#1}
% lazy way to have consistency in MoCA 2.0/2.5 band specification
\newcommand{\dband}{`D-Band' MoCA}
\newcommand{\xdband}{`Extended D-Band' MoCA}
\newcommand{\eband}{`E-Band' MoCA}
\newcommand{\fband}{`F-Band' MoCA}

%misc
\newcommand{\ngtv}{\textmdef{NextGenTV}}
\newcommand{\tdefcab}{\textdef{Telecommunications Access Box}}
\newcommand{\tcab}{\textmdef{Telecommunications Access Box}}
\newcommand{\atscrange}{\SIrange{54}{608}{\mega\hertz}}
\newcommand{\fmrange}{\SIrange{88}{108}{\mega\hertz}}

\usepackage[colorlinks=true,allcolors=black]{hyperref}
\title{Concerns about ATSC 3.0 Transition}

\author{Michael A. Peters}

% dev drafts - the date broadcast television first demonstrated
%  Update to current date for published PDF
%\date{September 7, 1927}
\InputIfFileExists{datehack.tex}{}{}

\begin{document}
\maketitle

\chapter*{abstract}

From the \myhref{https://www.fcc.gov/media/radio/public-and-broadcasting}{FCC Website}:

\begin{quotation}
\noindent
In exchange for obtaining a valuable license to operate a broadcast station using the public airwaves,
each radio and television licensee is required by law to operate its station in the ``public interest,
convenience and necessity.'' \hfill \\
\ldots \hfill \\
Station licensees, as the trustees of the public’s airwaves, must use the broadcast medium to serve the
public interest.  We at the FCC want you to become involved\ldots
\end{quotation}

During the first decade of this century, broadcast television in the United States transitioned from
analog broadcasts to digital broadcasts. This transition was in the best interest of the public, both
better picture quality and an increase of available content was made available as a result of the
superior compression available by using an encoded digital signal rather than an analog signal.

The transition did have a drawback: All consumers of broadcast television either needed to purchase
a new television capable of tuning \atsc{1.0} broadcasts, or alternatively purchase an external converter
box that hooked up to their existing television through an HDMI cable or RCA cables for cables that did
not have an HDMI interface, and some households in rural areas were no longer able to receive broadcast
television.

Some people were unable to afford the converter boxes. A subsidy program existed that helped these
people with a \$40 subsidy towards the converter box.

According to Nielsen Media Research. 3.1\% of Americans were not yet equipped for the transition to
digital broadcasts by the mandated cut-off date. I was actually one of those people, but I did not
sweat it because I was planning on purchasing a new television as soon as I could. I ended up needing
to wait until 2015 to get a new television, and thus went five years without television of any kind.

Now here we are just under a decade and a half since the last transition finished and we are
undergoing a new transition, the transition from \atsc{1.0} to \atsc{3.0}.

Like the transition from analog to \atsc{1.0}, there are definite technological benefits that serve
the public interest. However the transition from \atsc{1.0} to \atsc{3.0} is not just as simple as
buying a new television or a new converter box, many \atsc{3.0} broadcasts are encrypted and the
decryption of the signal \emph{requires} an active Internet connection that the tuner can use to
fetch the decryption key.

Particularly within the economically depressed population, there are many people who do not have
Internet service at their home. For those who do have Internet service at their home, they can
not always afford to pay the bill on time and sometimes they lose service. Even for those who
do have the financial means to pay their Internet bill on time, they may not have reliable service
that their television is capable of using. And finally, during disasters, it is not too uncommon
for Internet service to go out.

I do not believe it is in the `public interest' for the FCC to allow encrypted broadcasts on the
public airways. If the broadcasters are willing to continue broadcasting \atsc{1.0} along with
the encrypted \atsc{3.0} then \emph{maybe} it might be okay, but that is not what public broadcasters
wish to do. They want to terminate \atsc{1.0} broadcasts.

% stuff here

Appendix~\ref{apx:moca} (page~\pageref{apx:moca}) describes a possible update to the \ngtv{}
specification that would make it easier to get reliable Internet service to an \atsc{3.0}
tumer.

Appendix~\ref{apx:router} (page~\pageref{apx:router}) probably should be a separate document,
but it addresses changes in the Home Internet market that really need to be made to increase
the chances of the compromise suggested in Appendix~\ref{apx:moca} actually working.

\section*{PDF Version}
PDF versions of this document are available from \textit{update needed}. Two versions of the
PDF are made available: One version is compiled for double-sided printing on US Letter and the
other version is compiled for single-sided printing on US Letter. It is possible to modify
the source for printing on other paper sizes.

For members of the general public who read this, I ask that you print copies to send to your
government and to the FCC, preferably with your own cover letter. If they get a lot of copies
of this document sent to them from lots of different people, maybe just maybe they will actually
pay attention.

The PDF files are monochrome. Color printing is not required.


\section*{Hyperlink Note}

For typographical reasons, some hyperlinks had to be shortened using a URL shortener. Most URL
shorteners do in fact use tracking cookies. The domain used for the URL shortener is
\texttt{rb.gy}.

For privacy, you may wish to delete cookies in your browser that belong to that domain if you
follow any of those shortened hyperlinks.


\section*{Document Source}

The \LaTeX{} source for this document can be found at on
\myhref{https://github.com/YellowJacketLinux/atsc}{github}.


\tableofcontents

%\input{content/internet.tex}


\appendix

\chapter{A MoCA Compromise}
\label{apx:moca}

If the FCC decides to allow broadcasters to continue to broadcast encrypted \atsc{3.0}
then it is \textbf{imperative} that the FCC require
\ngtv{} certified tuners to implement \xdband{}~2.5 as a means by which to
assist households in achieving a stable Internet service at the \atsc{3.0} tuner.

It is my \emph{hope} that those behind \ngtv{} certification will see the wisdom in
implementing MoCA as part of the standard even without pressure from the FCC.

DirecTV `SWM' set-top boxes have built-in \eband{} clients (sometimes referred to as DECA),
Dish Network `Hopper and Joey' set-top boxes have built-in \fband{} clients, and many Cable
Television and TiVO set-top boxes have either built-in \dband{} or \xdband{} clients.

MoCA is an excellent networking solution when coaxial cables are involved. The Commercial
television delivery services that depend upon an Internet connection know this and have
implemented it, but I have not been able to find a single \atsc{1.0} or \atsc{3.0} tuner
that implements MoCA. That needs to change.

Many housing units in the United States of America already have coaxial cable infrastructure
installed and this infrastructure usually originates in a location close to the same
NEC compliant source of ground that an outdoor antenna is required by code to be bonded
to. This allows outdoor (or attic) antennas to connect to the distribution coaxial splitter
instead of Cable Television service, and also make it easy to run a MoCA bridge on that coaxial
network.

If all \ngtv{} tuners are \xdband{}~2.5 clients then providing them with stable
Internet access so they can retrieve decryption keys becomes as simple as connecting a
single \xdband{}~2.5 adapter to both the coaxial distribution splitter and to the
Internet router.

\ngtv{} tuners should use \moca{2.5} rather than \moca{2.0} so that users who want additional
security can enable \textmdef{MoCA Protected Setup} (aka MPS). \moca{3.0} can optionally
be implemented but would be wasteful on \atsc{3.0} tuners that have no need for the
bandwidth capabilities beyond \moca{2.5}. \xdband{} should be used because it does not collide
with the RF frequencies used by broadcast television.

As \moca{2.5} networks are limited to 16 nodes, \atsc{3.0} tuners should have the ability to
disable their built-in MoCA client \emph{even when a MoCA network is present} just in case
the household is using MoCA for lots of other things too and is running low on nodes.

\ngtv{} tuners should continue to offer WiFi connectivity on the \SI{2.4}{\giga\hertz} band
and the \SI{5}{\giga\hertz} band and ideally should also offer an Ethernet jack so that
Internet access can still be achieved in households where MoCA is not a workable solution---such
as households that use one indoor antenna per television and do not use a coaxial cable network.

If \ngtv{} requires MoCA clients be coupled with \atsc{3.0} tuners for certification, then the
odds of a tuner being capable of retrieving the decryption keys even when the tuner located in
a WiFi dead spot or problem spot within the home will increase significantly.

If someone can't change the channel because the microwave is on, it will be very annoying. That
sometimes happens with \SI{2.4}{\giga\hertz} WiFi but would not be an issue with MoCA.

If \ngtv{} is going to use an active Internet connection, then \xdband{} \textbf{MUST} be part
of the \ngtv{} standard.


\section{MoCA Retrofit Kit}

Some televisions with \atsc{3.0} tuners and some stand-alone \atsc{3.0} tuners that bear
a \ngtv{} certification are already on the market, and I suspect many more have
already been designed for the 2024 model year and it would be a financial burden to
forbid those products from hitting the market.

For such devices, I recommend a subsidized \textmdef{MoCA Retrofit Kit} be provided to
low-income households who have such devices. Unfortunately, this kit would be
useless for \atsc{3.0} tuners that lack an Ethernet jack.

Specific products mentioned below are simply products that I have personal experience with.
It is possible there are products from other manufacturers that are better suited for this
purpose.

The kit should include the following components:

\begin{description}
  \item[One Passive Two-Way MoCA Splitter] \hfill \\ I have personal experience with the
    \myhref{https://www.antronix.com/pdf/ds-1181-ss-a02-mmc1000-splitters.pdf}{Antronix MMC1002H-B}
    splitter to provide D-Band MoCA on `OTA Broadcast Television' coaxial networks. It works
    well.
  \item[One \xdband{}~2.5 Adapter] \hfill \\ I have personal experience with the
    \myhref{https://www.screenbeam.com/products/ecb6250/}{ScreenBeam ECB6250}
    MoCA 2.5 Network Adapters. That particular model has a \SI{1}{\giga\bit\per\second} Ethernet jack.
    They also make a more expensive model with a \SI{2.5}{\giga\bit\per\second} Ethernet jack
    (the ECB7250) but no \atsc{2.5} tuner needs that capability.
  \item[Two 18 Inch RG6 Coaxial Cables] \hfill \\
    One is needed to connect the television to the passive two-way splitter, one is needed to
    connect the MoCA Adapter to the passive two-way splitter.
  \item[One 3 foot Cat5e Patch Cable] \hfill \\
    Cat5e patch cables are far more flexible than Cat6 patch cables and are perfectly capable of
    handling the network speeds required of a MoCA 2.5 network.
\end{description}

In a nutshell, rather than having the antenna cable feeding the \atsc{3.0} tuner directly, it would
feed the two-way passive splitter. One split would then feed the \atsc{3.0} tuner, and the other
split would feed the \xdband{}~2.5 adapter which in turn would provide an Ethernet connection
to the \atsc{3.0} tuner so that the tuner, thus allowing decryption of encrypted \atsc{3.0}
broadcasts in cases where the tuner can not get a reliable WiFi connection.

\subsection{Cost Estimate}

Pricing those items (excluding the cables) at Amazon comes to about \$90.

That cost could probably be significantly reduced if purchased in bulk from the product manufacturers
but I do not \emph{personally} have the business experience needed to estimate such a cost reduction
and I would prefer not to speculate in a document I am sending to the government.

The ECB6250 as sold at Amazon comes with one coaxial cable and one Ethernet cable. The coaxial
cable that it comes with---to be frank---is junk. The cable is only dual-shield RG6 and thus susceptible
to RF interference from power cables which is likely when installed in the proximity of a television,
and the connectors on it are junk.

A 250 foot pull box of Mediabridge Quad-Shield RG6 costs about \$66 and would produce at least 150
\SI{18}{inch} cables, enough for 75 \textmdef{Moca Retrofit Kits}. That is 88¢ per kit.

A box of 50 quality Klein compression F connectors for quad-shield coax costs about \$45 and would
provide for 12 \textmdef{Moca Retrofit Kits}. That is \$3.75 per kit.

Ignoring labor, manually created quality RG6 cables would thus cost under \$5 per kit.

Again, I simply do not have the business experience to estimate how much that cost would be reduced
by ordering from a cable manufacturer that produces them in bulk by machine, but I would be very
surprised if the cost was not cut down by more than half. Quality quad-shield RG6 cables purchased in bulk
would not significantly add to the cost of the \textmdef{Moca Retrofit Kit} and would greatly decrease
interference problems that need to be solved post-install.

The Ethernet cable that the ECB6250 kit comes with is probably sufficient but quality patch cables
purchased in bulk are extremely cheap, they are heavily used in today's high-tech industry and are
mass-produced by numerous companies.

Three foot Cat5e patch cables purchased in quantities of 500 cables from
\myhref{https://rb.gy/s18bc}{CableWholesale} currently cost \$1.13 each and they do offer further
discounts for larger quantities.

\subsection{Fire Safety}

During my early twenties, I lived for a time near Home, Washington (Key Peninsula). My back caught
on fire. The fire was the result of inadequate safety that was a direct result of the poverty conditions
I was living under at the time.

In my case, the fire was the result of wearing a flanel shirt that had likely absorbed gasoline fumes from
power tools we were using to build a shelter, then ignited by me standing too close to a heat source on a
cold night when we did not yet have adequate shelter. I should have known better, but poverty sucks.
Fortunately one of the people living there was a former EMT who had lost his job and was homeless as a
result, so I did get the best possible first aid when it happened---and the EMTs in the ambulance and the
doctors were also top notch.

Ever since that incident though I have been paranoid about household fires, which seem to disproportionally
impact those living in poverty. As the proposed \textmdef{MoCA Retrofit Kit} is intended for those living in
poverty, fire safety should in my opinion be evaluated.

There are two fire safety concerns I have with the AC to DC transformer that comes with the previously
mentioned `ScreenBeam ECB6250' kit. Whether these concerns are genuine or not, I can not say, I may
be being overly paranoid.

The first concern I have is that I see no indication on the transformer itself that it has overcurrent
protection. It is true that Class~\rom{2} transformers are not capable of producing enough current to
electrocute someone but if there is a short in the device being powered, fire is still possible.

My second concern is that the transformer included with the `ScreenBeam ECB6250' kit is designed to fit
standard two-outlet wall receptable such that the body of the transformer does not block the use of
the outlet receptacle it is not plugged into.

When used for a television, that may actually be a problem. Most television areas have multiple items
that need to be plugged in, resulting in the use of a power bar. With most power bars, the outlet
receptacles are arranged such that the transformer body would block adjacent receptacles. This would
tempt the person to use a short extension cord between the transformer and the power bar, increasing
the odds of a loose connection causing an arc-fault.

The NEC has required arc-fault circuit breakers for most circuits in a home since 1999 but NEC is
not law and local code often lags behind the NEC. Furthermore, until 2007 or so (I might not be
correctly remembering) most arc-fault circuit breakers did not do a very good job at detecting
arc-faults in extensions to a circuit. Finally, those living in poverty are less likely to be
living in housing built after arc-fault protected circuits became code.

What I personally did in my current home with MoCA adapters, I used a
\myamazon{B00QV5Q5HU}{Ugreen} AC to DC transformer that has identical
electrical specifications (\SI{5}{\volt}, \SI{2}{\ampere}) with an identical positive-center
DC barrel (OD \SI{3.55}{\milli\meter}, ID \SI{1.35}{\milli\meter}) that has overcurrent protection
and has the transformer positioned such that it does not block adjacent outlet sockets on a
typical power bar.

Ideally, housing without arc-fault circuit breakers should install
\myamazon{B01CG8MP9W}{AFCI receptacles}
wherever power bars are used---cheaply manufactured power bars themselves are a common source of
arc-fault fires in addition to issues with cords not being properly plugged in---but that is beyond
the scope of the FCC or the Broadcast Television Industry.

\subsection{Signal Strength}
\label{apx:moca:signalstrength}

An unfortunate consequence of a passive two-way splitter is a reduction in the broadcast signal
strength. The actual signal attenuation that results varies by frequency but it will \emph{always}
be at least \SI{3}{\decibel} and the number Antronix prints on their two-way MoCA splitter is
\SI{3.5}{\decibel}.

When a \SI{3.5}{\decibel} loss in broadcast signal strength is an issue, it \emph{usually} can
be resolved by using a slightly more powerful amplifier, installing a better antenna, or a
combination of the two.

A technically better solution would be to use a band separator rather than a passive two-way
splitter.

As far as physical connections are concerned, a band separator is virtually identical to a
passive two-way splitter. The difference is with a band separator, RF frequencies below a
specific target are separated to one output and RF frequencies above that target are separated
to the other target.

An example of a band separator for coaxial cables is the
\myhref{https://pixelsatradio.com/products/am-fm-band-separator}{Pixel Technologies AMFMBS}.
That band separator is for radio broadcast bands and is not suitable for this application.

The reason why my recommended \textmdef{MoCA Retrofit Kit} specifies a passive two-way splitter
instead of a band separator is simply because I am not aware of an existing external coaxial
band separator that separates the ATSC Broadcast Band from \xdband{} and I suspect there
is not enough market demand to produce such a product at an affordable price. One certainly
could be produced, I just question if demand would warrant a supply production at an affordable
per-unit price.







%\input{apx/otabp.tex}
\chapter{Home Internet Service}
\label{apx:router}

My understanding is that `\ngtv{}' is a certification standard that manufacturers of \atsc{3.0}
tuners must build their devices to meet or else they do not get to call their product a \ngtv{}.

A similar standard is needed for home routers---to serve the best interest of the consumer.

This standard would \emph{especially} be needed if encrypted \atsc{3.0} broadcasts require an
active Internet connection to descrypt the stream (seriously, why can't caching the decryption
key be part of \ngtv{}???) but I believe it is needed even if the FCC does the right thing and
forbids encrypted broadcasts on public airways.

\section{Internet is a Utility, not a Luxury}

For decades now, the model for home Internet access that has been pushed by both Internet
Service Providers and the consumer market has been routers that include WiFi antennas---and now
the trend is getting even worse, now to get `whole house' coverage the consumer has vendor lockin
and often has to buy (or rent) `Mesh WiFi' extenders that are the same brand and WiFi version as
the ISP provided router + WiFi. Vendor lockin is \emph{always} bad for the consumer.

Internet service needs to be viewed in a similar fashion as a utility.

With the Electrical utility, the power main and everything behind the power main is part of the
housing unit. The Electrical utility provides power to the power main, but the main and everything
behind it is part of the home.

The same can be said for the gas utility, water utility, even the Cable Television utility. Note that
with Cable Television, the industry wanted to force their proprietary boxes on customers but
Telecommunications Act of 1996 required them to allow `Cable Cards' to be used in hardware that
was owned by the consumer.

Also note that in the days of `Ma Bell', the phone company owned all the wiring in the home and
even owned the telephones in the home. Federal law changed that too, housing had to have a
demarcation point (usually a punch-down block or an NID) and everything downstream of that
demarcation point was part of the home, the telephone company simply provided telephone service
to that demarcation point.

Internet service needs to be the same way. The `Home Network'---a type of LAN in technical terms---should
be seen as part of the home. The ISP should be seen as a utility provides Internet service to the `Home Network'.

With some broadband services---such as Fiber Optic and Cable Modem service---that frequently is
the way things are, but there are two growing trend underming that model \emph{especially} for
low-income households, and the result is an insufficient LAN that provides inferior service.

\section{The Home Local Area Network}

The home \textdef{Local Area Network (LAN)} consists of the router and everything `behind' the router
whether it is \textmdef{wired} (Ethernet or MoCA) or \textmdef{wireless} (WiFi).

I am actually not opposed to Internet Service Providers opting to provide service that does not meet
the \textdef{Ideal LAN} described below, but they \emph{should also} offer service that is compatible
with the \textmdef{Ideal LAN} described below.

\begin{itemize}
  \item An \textmdef{Ideal LAN} needs to be independent of the service provided by the Internet
Service Provider.
  \item Any components of the LAN---including the router---that are provided by the ISP must be either
        purchased by the consumer outright \emph{at a fair market value} or at the discretion of the ISP,
        they may be provided on a `rent to own' program. The current scenario where many customers
        pay monthly `equipment fee' that pays for the equipment the ISP technically still owns many times over
        needs to stop. The total `rent to own' cost of a component provided by the ISP should never exceed
        140\% of the fair market value of the equipment.
  \item Such \textmdef{Ideal LAN} equipment \textbf{MUST} be vendor independent equipment. If the consumer
        decides they want to switch to a different ISP, the consumer needs to be able to do so \emph{without}
        needing to replace any of the hardware in an \textmdef{Ideal LAN}.
  \item Hardware \emph{between} the router and the ISP, such as a modem or ONT, are NOT to be considered
        part of the \textmdef{Ideal LAN} and may be proprietary and ISP owned.
\end{itemize}

Some Internet Service Providers---such as T-Mobile and Verizon---currently offer 5G Broadband Internet
service using a single piece of hardware that connects to their 5G service and only provides Internet
via WiFi.

For some consumers---especially those in small apartments---that genuinely may be what they want and
those ISPs shouls be allowed to continue to offer those services.

However, any customer should be able to tell them they want the Internet Service but instead of that
WiFi only router, they want a modem that connects to an \textmdef{Ideal Local Area Network}.

In the current age of `Internet of Things' where WiFi is necessary for the proper operation of many
household appliances, the `Router plus Access Point in one device' unit has led to the proliferation
of `WiFi Extenders' and `Mesh Networks' which I believe is \emph{rarely} a good solution and frequently
degrades Internet service in the home.

The consumer \textbf{MUST} have the inherent right to design their home network in such a way that
it properly meets their Internet connectivity needs without limiting their choice in Internet Service
Providers and without resulting in vendor lock-in to a specific Internet Service Provider.

To accomplish this, there must be a router specification similar in purpose to the \ngtv{} specification
such that when a consumer buys a router that meets that specification, the consumer knows they can
build a LAN around that router without any limitation of Internet Service Providers they can choose
from. That is what this appendix chapter is about.

\section{The Ideal Router}

What I am describing here is something that I call the \textdef{Ideal Router}. It is not the right device
for every `Home Internet' scenario but I do believe it is the right device for the vast majority of
`Home Internet' scenarios, which would allow for mass production and thus a low price.

Many homes have what I call a \tdefcab{} which is a box often inset inside a wall that frequently houses
the following low-voltage signal related wiring related items. Its typical contents follow.

\pagebreak

\begin{itemize}
  \item For homes with Ufer grounding, a fat copper wire that is bonded to the Concrete Encased Electrode
        and can be used by coaxial networks and telephone service as an NEC compliant source of ground.
  \item For homes with `Plain Old Telephone Service', a Telecom NID (Network Interface Device) or a telephone
        wiring punch-down block.
  \item The entry point for the Cable Company coaxial cable into the premises.
  \item The entry point for Satellite Television coaxial cable into the premises.
  \item The entry point for Television and/or Radio antenna coaxial cable into the premises.
  \item Any needed coaxial signal amplifiers.
  \item Any needed coaxial signal distribution splitters.
  \item The entry point for a Fiber Optic run into the premises.
\end{itemize}

Sometimes, the \tcab{} is actually two (or more) boxes with the secondary box typically in the basement or a garage.
Any secondary box will not have a NEC compliant ground source as it is too far away from the power main, but they
will have low-voltage signal wires coming from the first box. The concept is that safety grounding and sometimes
signal amplifications takes place in the first box but the signal distribution may take place elsewhere.

Nutshell: \tcab{} as used in this document may be one location but could encompass other locations with wiring
leading back to one location.

The idea behind the \textmdef{Ideal Router} is a small form factor router that can optionally be installed
\emph{inside} the \tcab{} along with the source of broadband Internet---so that it is physically close to
the wiring source of the broadband Internet.

Preliminary specifications for the \textmdef{Ideal Router} follow on the next page.

\clearpage

\begin{description}
  \item[USB Modem Port] \hfill \\ The router \emph{must} have a USB 3.1 (or better) port that can optionally
       be used for a USB modem.
  \item[RJ45 WAN Port] \hfill \\ The router \emph{must} have a single \SI{1}{\giga\bit\per\second} RJ45 WAN port
       that is capable of pass-through PoE to power a PoE DOCSIS modem or a PoE ONT (Optical Network Terminal).
       That would reduce the odds of needing a \SI{120}{\vrms} power socket inside the Telecommunications Access
       Box. The router RJ45 WAN port \emph{may} support speeds beyond \SI{1}{\giga\bit\per\second}.
  \item[RJ45 LAN Port] \hfill \\ The router \emph{must} have a single \SI{1}{\giga\bit\per\second} RJ45 LAN port
       that is capable of powering the router via PoE. The router RJ45 LAN port \emph{may} support speeds beyond
       \SI{1}{\giga\bit\per\second} and \emph{may} have additional RJ45 LAN ports.
  \item[Coaxial MoCA LAN Port] \hfill \\ The router \emph{must} have a single \SI{75}{\ohm} F-jack port that
       provides an `Extended D-Band' \moca{2.5} uplink to the LAN. The router \emph{may} implement \moca{3.0}
       as long as it maintains compatibility with \moca{2.0} and \moca{2.5}.
  \item[USB-PD Port] \hfill \\ The router \emph{should} have a single `USB-PD' port that can be used to power the
       router if PoE is not used to power the router.
  \item[OpenWrt] \hfill \\ The router \emph{must} be able to run \myhref{https://openwrt.org/}{OpenWrt} as the router
       operating system but it \emph{may} ship with another operating system. As an open-source operating system,
       OpenWrt reduces the odds of a perfectly functional router from needing to be replaced simply because the
       manufacturer went out of business or deprecates the model, no longer responding to security vulnerabilities.
  \item[No Nameserver] \hfill \\ The router \emph{must not} have a nameserver. It is not a bad thing to have a
       caching recursive nameserver on the LAN, but if present it should be on a physically different piece of
       hardware than the router for security reasons.
  \item[No WiFi] \hfill \\ The router \emph{must not} have WiFi antennas.
\end{description}

Those are my suggested specifications. If the `\textmdef{Ideal Router}' (under that name or another name) were to
become a standard for home networking, obviously there would have to be discussions about the specification.

Hopefully an `Open Hardware' reference board could be designed that any manufacturer could use as a starting point
(or just produce as-is).

The sections that follow describe how such an \textmdef{Ideal Router} should be used in a home to provide wired LAN
service and WiFi service, and how the router would connect to an Internet Service Provider.

Following the suggested network topology sections are additional sections discussing potential implementation issues.

\section{LAN Distribution Switch}

When a home has been wired with Cat5e or better twisted-pair network cables, a network switch will be needed to
provide LAN access to the home through those network cables.

This switch is what I call the \textdef{LAN Distribution Switch} and should be located in the vicinity of where
those network cables originate.

When possible, the \textmdef{LAN Distribution Switch} should have a Cat6 uplink to the `RJ45 LAN Port' on the
\textmdef{Ideal Router}. This uplink cable should be \SIrange{23}{24}{\awg} solid copper so that it properly
supports Power over Ethernet. In other words, do not use one of those nifty `flat' Cat6 cables for this purpose,
they tend to only be \SI{30}{\awg} and will overheat and potentially melt when carrying DC current.

When possible, the \textmdef{LAN Distribution Switch} should support the PoE specification sufficient to
power the \textmdef{Ideal Router} and any Wireless Access Points connected to it that are powered by PoE. A proper
PoE injector could also be used.

When there is not Cat6 uplink from the \textmdef{LAN Distribution Switch} location to the \textmdef{Ideal Router}
location, a \moca{2.5} uplink does work very well, I have done this quite successfully in the past.

The power source for the \textmdef{LAN Distribution Switch} (and a PoE injector if used) ideally would be plugged
into an adequate UPS so that the \textmdef{Ideal Router} does not need to reboot every time there is a
momentary loss of power---something that happens often in California during the summer as some mis-behaving
transformers are powered off due to fire conditions.

\section{Extended D-Band MoCA LAN}

The coaxial cable network will need an \xdband{} compatible \textdef{Distribution Splitter} to distribute the
coaxial signal to the destination jacks. To be \xdband{} compatible, it needs to support RF frequencies up to
\SI{1675}{\mega\hertz}.

In some cases due to space or other constraints, the \textmdef{Distribution Splitter} may actually be a two- or
three-way splitter that feeds additional splitters that then feed the coaxial cable runs to the final
destinations.

The \textmdef{Distribution Splitter} should be a \emph{passive} splitter. They do make amplified splitters that
work with \xdband{} but they can be problematic and should generally be avoided.

Any needed amplification of the RF signals coming from an ATSC antenna or from the Cable Television service should
take place \emph{between} the signal source and the \textmdef{Distribution Splitter}.

A power-passing \textmdef{Distribution Splitter} should not be used---or at least not used to pass power.
Any amplification that uses DC power transmitted over a coaxial cable should have a coaxial PI (e.g.\
the \myhref{https://www.antronix.com/pdf/DS-1020-SS-A04_ARPI-2000.pdf}{Antronix ARPI-2000}) \emph{between} the
signal amplifier and the \textmdef{Distribution Splitter}.

When a coaxial Power Inserter (PI) is used, there also should be a coaxial surge protector between the PI
and the \textmdef{Distribution Splitter} even if there is already a coaxial surge protector between the
signal source and the signal amplifier.

The purpose of this secondary surge protector is to protect the passive coaxial network from surges on the
coaxial core that enter through the PI and/or amplifier.

This secondary surge protector \emph{should not} have a bonded ground wire as the coaxial shield on the
coaxial network should \emph{already} be properly grounded according to NEC specifications where the
coaxial cable enters the building structure. Multiple paths to ground is \textbf{not} a good thing, it
can introduce ground loops that cause RF interference.

The \xdband{} compatible splitters that I have extensive \emph{successful} experience with are the Antronix
\myhref{https://www.antronix.com/pdf/ds-1181-ss-a02-mmc1000-splitters.pdf}{MMC1000-B Series} splitters.

At the input of the \textmdef{Distribution Splitter}, there should be a
\myhref{https://www.antronix.com/products/filters}{MoCA `Point of Entry' Filter}. For \xdband{},
the `Point of Entry' filter is basically just a \SI{1002}{\mega\hertz} low-pass filter.

With the coaxial network now properly prepared for MoCA, run a \SI{75}{\ohm} coaxial cable from the
`Coaxial MoCA LAN Port' on the \textmdef{Ideal Router} to the an available port on the
\textmdef{Distribution Splitter} and any coaxial jack downstream of the \textmdef{Distribution Splitter}
is a potential \xdband{} MoCA node.

If every jack on the \textmdef{Distribution Splitter} is already in use, you may need to get creative. For example,
if there is an 8-way \textmdef{Distribution Splitter} and eight coaxial drops originating from it, one solution would
be to use a
\myhref{https://antronix.myshopify.com/collections/antronix-moca-splitters/products/mmc1003bh-b}{MMC1003BH-B}
balanced three-way splitter. Two of the outputs would each go to a
\myhref{https://antronix.myshopify.com/collections/antronix-moca-splitters/products/mmc1004h-b}{MMC1004H-B}
four-way splitter. The third output would be used to connect to the `Coaxial MoCA LAN Port' on the
\textmdef{Ideal Router}.


\subsection{MoCA Only Coaxial Networks}

When the coaxial network is \emph{only} used for MoCA and is not used for either Cable Television or for
Broadcast Television, then a MoCA `Point of Entry' filter is not needed at the input of the \textmdef{Distribution Splitter}.
Instead, use the input of the \textmdef{Distribution Splitter} as the connection point for the coaxial connection from the
`Coaxial MoCA LAN Port' on the \textmdef{Ideal Router}.

Note that in this MoCA only scenario, the coaxial network will not be grounded either by the Cable Company
demarcation or by the ATSC antenna coaxial grounding. It thus would be a good idea to connect a \SI{10}{\awg}
bonding wire from the \textmdef{Distribution Splitter} to an NEC compliant source of ground just to provide better
EMF shielding to the coaxial network.

The next two subsections relate to MoCA on DirecTV and Dish Network and are not related to \xdband{}
from the \textmdef{Ideal Router}. If you do not care about these alternative MoCA implementations, you
can skip to section~\ref{apx:router:wifi} on page~\pageref{apx:router:wifi}.

\subsection{DirecTV Note}

My only experience with DirecTV is with single-household SWM installs.

A DirecTV \textmdef{Distribution Splitter} that is used for single household SWM installs is a passive
splitter but one port on the splitter is power-passing. If you (or the installer, they do not generally
allow customer self-installs) are going to use that port to power the dish then the coaxial cable coming
from that port \emph{must} be solid-copper core \emph{and} there are distance limitations. The coaxial
cables installed by a home-builder are usually copper-clad steel core, which is fine for signal transmission
but is not suitable for DC power transmission (too much DC resistance).

When the coaxial cable drops in your home are copper-clad steel, do not let the installer tell you they
have to run new RG6. There are two other solutions:

\begin{enumerate}
  \item You can put the DirecTV SWM Power Inserter between the dish and the distribution splitter. Their
        power inserter is big and bulky as the transformer is part of it, and it requires access to a
        \SI{120}{\vrms} socket. When that is not practical:
  \item Note that I have not personally tried this. You can use a Holland Electric HRF-PI---which is
        very compact---between the dish and the distribution splitter. You will need a run of
        solid-copper core RG6 from the HRF-PI to a location where the DirecTV SWM Power Inserter can
        be plugged in. Installers might refuse this option as the HRF-PI is not DirecTV approved, but
        it \emph{should} work. The `DC+RF' port would connect to the dish (and must be solid-copper core
        RG6). The `RF' port would connect to the input of the DirecTV SWM \textmdef{Distribution Splitter}
        and can be copper-clad steel RG6. The `DC' port would connect to the DirecTV SWM Power Inserter
        and would need to be solid-copper core RG6. The DirecTV SWM Power Inserter would no longer be
        able to have a receiver attached, it now only provides power. All DirecTV receivers have to be
        downstream of the DirecTV SWM \textmdef{Distribution Splitter} and those runs can be copper-clad
        steel RG6.
\end{enumerate}

For MoCA, DirecTV SWM receivers have built-in MoCA clients that use the `E-Band' (\SIrange{400}{700}{\mega\hertz}).
This is often called `DECA' by users but I am under the impression that `DECA' technically only refers to the
stand-alone \eband{} adapter that DirecTV manufactures.

The \xdband{} adapter in the \textmdef{Ideal Router} would not be usable for establishing a DirecTV MoCA
network. Tips on how to get MoCA working without using the \textmdef{Ideal Router} follow.

With DirecTV, you can not use the `Point of Entry' filter described earlier.

DirecTV does make a unit called a `DECA Band Stop Filter'
(\myhref{https://www.solidsignal.com/directv-band-stop-filter-for-multi-room-viewing-systems-bsfr01}{BSFR01})
but it seems that with SWM systems it is not needed for single-housing unit installs.

On multi-housing unit installs when a single dish serves multiple residences, each residence served by
the dish must have its own DirecTV SWM \textmdef{Distribution Splitter} and each DirecTV SWM
\textmdef{Distribution Splitter} \emph{must} have a BSFR01 installed in the feed to that DirecTV SWM
\textmdef{Distribution Splitter} to prevent MoCA networks from different housing units interfering with
each other.

If you live in an apartment with DirecTV service and you can see content on a neighbor's DVR, that `DECA
Band Stop Filter' is missing.

I am under the \emph{possibly incorrect} impression that with some DirecTV SWM DVR receivers, if you can
connect the Ethernet jack on the DVR receiver to a working Ethernet jack with network connectivity,
all SWM DirecTV clients will then be able to use the DirecTV MoCA network to get Internet access.

Another option I \emph{know} works is to buy a
\myhref{https://www.solidsignal.com/DIRECTV-GigE-DECA-ATT-Network-Adpater-with-Power-Supply-and-Cable-DCAGE11KIT}{DIRECTV
GigE DECA Kit}.

With that kit, you need to run an RG6 coaxial cable from the adapter to the DirecTV SWM \textmdef{Distribution Splitter}
and a Cat6 cable from the adapter to a working Ethernet jack that has network connectivity. Then
all DirecTV SWM receivers on the network will be able to get Internet access through \eband{}.

\subsection{Dish Network Note}

\textbf{Disclaimer:} I have no personal experience with Dish Network.

My understanding is that Dish Network uses the MoCA `F-Band' (\SIrange{650}{875}{\mega\hertz}) and that if your Hopper
receiver has an Ethernet connection through its Ethernet jack, it will provide Internet through MoCA to all of
the Joey receivers on the network.

I know nothing about how to provide MoCA to a Dish Network coaxial network if the Hopper is in a location without
wired Internet access. I can not seem to find any stand-alone \fband{} adapters on the market.



\section{WiFi LAN}
\label{apx:router:wifi}

The \textmdef{Ideal Router} does not provide WiFi services itself. Rather, separate `WiFi Access Points'
should be used that uplink to the \textmdef{Ideal Router}.

Optimal places within the house to place Wireless Access Points should be determined, preferably with the
help of someone who has experience with WiFi topology. The
\myhref{https://www.reddit.com/r/HomeNetworking/}{HomeNetworking subreddit} can be extremely helpful for
figuring out an ideal WAP topology for a home.

Locating and using the optimal locations for Wireless Access Points is a critical step. Many problems that
people have with WiFi would not exist if they had WAPs located in proper places for whole-house coverage,
taking into consideration things like high-use areas where users are likely to want to sit and use their
phone, table, or laptop.

If a location for a WAP is close to a twisted-pair Ethernet jack, the WAP should uplink to the
\textmdef{LAN Distribution Switch} through that Ethernet jack, using a supplemental network switch between
the WAP and that Ethernet jack if more than one wired Ethernet device needs to uplink through that jack.

If there is not a twisted-pair Ethernet jack but there is a coaxial jack, the WAP should uplink over MoCA
using an \xdband{} adapter that is connected to that coaxial jack. Those with DirecTV should use a DECA
\eband{} adapter instead but be warned the bandwidth is not as good.

If more than one wired Ethernet device needs to uplink through that MoCA adapter, then again, an inexpensive
Ethernet switch can be used between the various Ethernet devices and the MoCA adapter.

Note that some WAPs are now capable of speeds beyond \SI{1}{\giga\bit\per\second}. If those access points
have a \SI{2.5}{\giga\bit\per\second} Ethernet jack and is uplinking through MoCA, then it may be worth the
extra money to buy a \moca{2.5} adapter that also has a \SI{2.5}{\giga\bit\per\second} Ethernet jack.

If the coaxial jack is also needed for a television or Cable set-top box, you can use an \xdband{} compatible
passive two-way splitter to provide both a MoCA connection and a television connection.

In some cases using a passive two-way splitter can cause signal reception issues, see
subsection~\ref{apx:moca:signalstrength} on page~\pageref{apx:moca:signalstrength}.

\subsection{WiFi Bands}

If you do not like to waste money, be careful about determining what WiFi bands you actually need and what
WiFi bands the capitalist marketing machine tells you that you need.

\wifi{3} (\ieee{802.11g}), \wifi{4} (\ieee{802.11n}), \wifi{5} (\ieee{802.11ac}), and \wifi{6} (\ieee{802.11ax})
prior to \wifi{6e} support WiFi operation on both the \SI{2.4}{\giga\hertz} and \SI{5}{\giga\hertz} bands.
These WiFi Access Points are often called `Dual-Band' WiFi Access Points.

The \SI{2.4}{\giga\hertz} band has a lower data transfer rate but has a much longer range, making it ideal
for most IoT devices but not as ideal for high data consumption applications, like downloading all the
video advertisements on a web page.

The \SI{5}{\giga\hertz} band has a much higher data transfer rate but with a much shorter range, making it
less ideal for IoT devices that do not need a high data transfer rate but much \emph{much} better for
devices that do need a high data transfer rate.

\wifi{6e} (\ieee{802.11ax} extended) and \wifi{7} (\ieee{802.11be}) also support the \SI{6}{\giga\hertz} band.

The biggest benefit of the \SI{6}{\giga\hertz} band is that it reduces the congestion of devices fighting
for bandwidth from the \SI{5}{\giga\hertz} band. This allows more high-data consumption devices to connect
at the same time (some using \SI{5}{\giga\hertz}, others \SI{6}{\giga\hertz}) and also makes it easier to
allocate \SI{160}{\mega\hertz} wide channels when \emph{very} high data consumption is genuinely needed
but a wired network is not available.

I do not believe the vast majority of households will see any benefit from \wifi{6e} or \wifi{7} that they
would not also see from less expensive \wifi{6} \emph{without} the \SI{6}{\giga\hertz} band. The vast majority
of households simply do not have the congestion on the \SI{5}{\giga\hertz} band to justify an additional band,
\emph{especially} if stationary high data consuming devices like gaming consoles can get their uplink to the
router through MoCA and/or standard Ethernet.

It can be argued---and often is argued---that a third band is beneficial to households by providing a
band for a mesh network. That is discussed next.

\subsection{Mesh WiFi}

If there is neither a twisted-pair network jack nor a MoCA coaxial jack in the vicinity of where a WAP is
needed \emph{and} running a network cable is not a possibility, then a Mesh WiFi system may in fact be
needed.

This should however be a \emph{last} resort solution.

The way Mesh WiFi works, the WAPs use three bands. In addition to the standard \SI{2.4}{\giga\hertz} and
\SI{5}{\giga\hertz} bands that the access points offer to clients to connect, they have a third antenna
that either also uses the \SI{5}{\giga\hertz} band or in recent systems, the \SI{6}{\giga\hertz} band.

This third band is used for the Mesh WAPs to communicate with each other.

When a Mesh WAP has a wired (Ethernet or MoCA) uplink, it sends all traffic from its clients through
that uplink just like a standard WAP would. However, it also is now responsible for all the traffic from
other Mesh WAP units that do not have a wired uplink but are sending their traffic to it over the third band.
This potentially reduces the network throughput for its clients.

The farther away two Mesh WAPs that connect to each other are from each other, the slower the communication
between them becomes. Sometimes the data speed between a Mesh WAP without a wired uplink and a Mesh WAP with
a wired uplink is so bad that an additional Mesh WAP needs to be placed between them solely for the purpose
of increasing the transmission speed.

It gets even worse. WiFi clients have no way of knowing which access points have a wired uplink and which
access points do not, they only look at signal strength.

A WiFi client in the range of two Mesh WAPs may choose to connect to the Mesh WAP without a wired uplink
just because it has a stronger signal strength even though all of its traffic ultimately goes through
the other WAP anyway, with slower service speed to the client than if it had just connected to the WAP with
the weaker signal strength.

Home Networking should avoid Mesh WiFi whenever possible. Google and some other companies have spent a
lot of money turning Mesh WiFi into a buzzword that makes it sound like this really cool modern way of
spreading WiFi throughout the home, but they have a financial motive. They make a lot of money when
you buy their expensive Mesh hardware.

Standard Wireless Access Points using either Ethernet or using a MoCA network to
uplink to the router is a \emph{much} better solution for whole-house WiFi.

\section{Fiber Optic WAN}

Whenever Fiber Optic service is available, it should be given strong weight over a Cable Modem as the
Internet service to use. It is fast and Fiber Optic service does not use a metal conductor,
thus eliminating one of the mechanism by which an equipment damaging and potential fire starting
electrical surge can enter the premises.

Fiber Optic Internet Service also frequently has much better upload performance than similarly priced
DOCSIS Cable Modem service.

With my \textmdef{Ideal Router}, all that would be needed is a Cat6 patch cable from the ONT `LAN'
port to the router `WAN' port. If both the router and the ONT are housed in the `Telecommunication
Access Box' an inexpensive off-the-shelf patch cable should do it.

I am \emph{personally} only aware of one ONT that can be powered by PoE, the
\myhref{https://dl.ubnt.com/guides/ufiber/UF-LOCO_QSG.pdf}{UFiber Loco}. If my \textmdef{Ideal Router}
does get produced, hopefully more PoE powered ONTs will hit the market.

\section{DOCSIS Cable Modem WAN}

DOCSIS is the protocol by which Cable companies provide Internet service using the same coaxial cable
by which they also provide Cable Television service.

When a DOCSIS Cable Modem is used, it involves a coaxial cable from a Cable Company.

When installed according to electrical code, the coaxial cable from the cable company will connect to a coaxial
grounding block that has a \SI{10}{\awg} or larger bond wire properly connected to a electrical code
approved source of ground. This grounding block is often (at least by me) referred to as the
\textdef{Coax Demarcation Point}.

The NEC does \emph{not} require a coaxial surge suppressor be used, and at least in California, it
seems the vast majority of Cable Service companies opt not to install one when they set up cable service.

It is my \emph{strong opinion} that a coaxial surge suppressor \textbf{SHOULD} be used. I would encourage
people to read the Tii Technologies
``\myhref{https://www.tiitech.com/pdfs/Tii_surge_protection.pdf}{Surge Protection's Essential Role in
the Cable Telecommunications Triple Play}'' document.

It is my \emph{strong opinion} that a proper \textmdef{Coax Demarcation Point} should use a
\myhref{https://tiitech.com/wp-content/uploads/datasheets/coax-protectors/210_212.pdf}{Tii 212FF75F225-31}
(or equivalent) as its grounding block.

Now that the \textmdef{Coax Demarcation Point} has been defined, there are typically two scenarios under
which a DOCSIS Cable Modem would be used that impact the house low-voltage wiring:

\begin{enumerate}
  \item Cable Modem service without Cable Television
  \item Cable Modem service with Cable Television
\end{enumerate}

Under the first scenario, a coaxial cable would run from the \textmdef{Coax Demarcation Point} directly to a
DOCSIS cable modem---perhaps with an inline attenuator if the signal is too hot.

Under the second scenario, a passive two-way splitter is needed at the \textmdef{Coax Demarcation Point}.
For DOCSIS modems prior to \docsis{3.1}, an Antronix
\myhref{https://antronix.myshopify.com/products/cmc2002ha}{CMC2002HA} (or equivalent \SI{1002}{\mega\hertz})
passive splitter should be used.

For \docsis{3.1} `Phase 1', an Antronix
\myhref{https://antronix.myshopify.com/products/cmc4002h}{CMC4002H} (or equivalent \SI{1218}{\mega\hertz})
passive splitter should be used.

For \docsis{3.1} `Phase 2' or \docsis{4.0}, an Antronix
\myhref{https://rb.gy/542gd}{CMC5002H-A} (or equivalent \SI{1800}{\mega\hertz})
passive splitter should be used. Note that at the time of writing, these splitters do not yet seem to be
available to the general public.

Regardless of the two-way passive splitter used, one output would have a coaxial cable that leads directly
to a DOCSIS cable modem and the other output would have a coaxial cable that leads to the MoCA `Point of
Entry' filter attached to the input of \xdband{} compatible \textmdef{Distribution Splitter}.

I would like to explain that the reason for a passive two-way splitter coming off the \textmdef{Coax Demarcation
Point} is to avoid a collision between the RF frequencies used by DOCSIS and \xdband{}. This
collision does not happen with DOCSIS versions prior to \docsis{3.1} but those earlier DOCSIS versions are
being phased out, so the cable infrastructure should be designed around the collision being a factor regardless
of the DOCSIS version.

For the \textmdef{Ideal Router} to uplink through the DOCSIS modem, there would need to be a Cat6 cable from the
DOCSIS modem to the `RJ45 WAN Port' of the \textmdef{Ideal Router}.

I am not personally aware of any DOCSIS modems that can be powered by PoE but hopefully the industry will
start producing some. If the DOCSIS modem is powered by PoE, then the Cat6 cable should be \SIrange{23}{24}{\awg}
solid copper so that it properly supports Power over Ethernet.

\section{USB Modem WAN}

\textbf{Disclaimer:} I have no personal experience with USB modems.

A USB modem is a modem that provides Internet service through a USB interface. Historically they typically
used the 3G or 4GLTE network and were rather disappointing, but now some 5G devices are starting to
appear---like the \myhref{https://www.tricascadeinc.com/5g-dongle}{VOS 5G Network Adapter} that uses T-Mobile's
5G network---that look quite promising.

USB modems that use uplink connections other than over a cellular network (such as dial-up over POTS) also
exist.

When either Fiber Optic or DOCSIS service is available, those options \emph{probably} should be preferred
for household Internet. When they are not available, or when they are not affordable, or when a `backup ISP'
is desired, then a USB modem that uses the cellular network is often a good option.

For those who do not have bank accounts, Fiber Optic or DOCSIS service may not be an option because the
providers often want direct billing, but prepaid service with USB modems does exist.

On an \textmdef{Ideal Router}, a USB Modem would connect to the `USB Modem Port'.

\subsection{Backup ISP}

When both the `RJ45 WAN Port' and the `USB Modem Port' on an \textmdef{Ideal Router} are connected, the
router software should allow selection of which interface is the \emph{primary} ISP uplink and which interface
should be used as the \emph{backup} ISP uplink.

The VOS 5G USB modem seem to allow you to pay a small monthly fee for a small data cap but does
not cut off service if you exceed that cap, instead just greatly reducing your speed. That actually makes
them ideal as a backup service because if your main ISP is out long enough that you do hit the data cap---you
still have service, just slower, but it would probably work just fine for retrieving the decryption key
that an encrypted \atsc{3} broadcast would need.

The VOS 5G however (allegedly, I have not bought one) requires a credit check. The impression I got was
that the credit check is required by T-Mobile.

Seriously, a credit check for a device that is purchased and then service is paid monthly? I need confirmation
on that, but if true, that should be illegal.

When I hear about credit checks being required when neither money nor property is being borrowed---I get
\emph{extremely} angry. Classism in providing quality Internet Service is something that needs federal
attention.

The monthly plans for the T-Mobile service required by a `VOS 5G USB modem' are inexpensive enough they
should be a means by which someone on the `wrong side of the tracks' can \emph{build} their credit, they
should not be something that is restricted to those who already have sufficient credit.

When those of us with some admittedly socialist leanings say we want essential utilities like Internet
service to be public owned utilities, the blatant discrimination of the private capitalist market is
exactly why.

\section{Ideal Router Close Products}

I am not \emph{personally} aware of any routers that currently meet my specification but I did find a few
that are not too far off.

\subsection{ER605v2}

The \myhref{https://www.tp-link.com/us/business-networking/omada-sdn-router/er605/}{TP-Link ER605v2} is an
existing product that is \emph{close} to the \textmdef{Ideal Router} specification, lacking the following
features:

\begin{itemize}
  \item No PoE support. It must be powered by a transformer brick and can not power a PoE WAN device.
  \item Limited USB Modem support. The USB jack is a USB~2.0 Type A jack. The 5G USB modems that I have seen
        require USB~3.1 or newer.
  \item No \xdband{} support. A separate \xdband{} adapter that uplinks to the wired LAN
        is needed.
  \item No OpenWrt support, but it \emph{appears} that is being worked on.
\end{itemize}

The ER605v2 also has more RJ45 Ethernet jacks than I would like in the \textmdef{Ideal Router}, one WAN and
one LAN would help keep the device small which is an important consideration when the install location is
inside a \tcab{}. Using a \textmdef{distribution switch}---even just an unmanaged
Layer 2 switch---to provide all the needed wired Ethernet jacks is a pretty standard practice that helps
reduce the load on the router.

A second RJ45 WAN jack is useful for some people for a backup Internet connection but I really think that
a USB modem providing that is sufficient.

Nothing would stop manufacturers who think a second WAN and/or second LAN is beneficial from adding the
capability.

\subsection{G3100}

The \myamazon{B082BLM2KH}{Verizon/Fios G3100} is an existing product that offers \moca{2.5} but has the
following failures:

\begin{itemize}
  \item No PoE support. It must be powered by a transformer brick and can not power a PoE WAN device.
  \item Limited USB Modem support. The USB jack is a USB~3.0 Type A jack. The 5G USB modems that I have seen
        require USB~3.1 or newer.
  \item It is big and bulky and would not fit in a \tcab{}.
  \item It has built-in WiFi. WiFi should be separate.
  \item Even worse, the built-in WiFi supports WPS. WPS is insecure and should be considered obsolete
        technology that is \emph{never} supported in new hardware.
  \item It has a caching nameserver. A nameserver should never be on the same physical hardware as the
        router.
  \item No OpenWrt support.
\end{itemize}

Ignoring PoE, if you take away the built-in WiFi---thus reducing the size---and the nameserver, update the USB port,
add OpenWrt support, then the G3100 could be an \textmdef{Ideal Router}. It does have more RJ45 jacks than I like but
that is not a hill to die on.


\section{Potential `Ideal Router' Issues}

The location of the \tcab{} sometime is such that the temperature inside it
may exceed the temperature where it is safe to operate routers, modems, and ONTs.

In such cases, temperature sensitive equipment---including the \textmdef{Ideal Router}---will need to be
located where it is cool enough for them to safely operate, with the appropriate wires running back to the
\tcab{}.

Standard PoE (IEEE~802.3af) has a limit of about \SI{15}{\watt}. That may not be enough to power the
\textmdef{Ideal Router} if the router is in turn also powering a DOCSIS modem or ONT. However, PoE+ (IEEE~802.3at)
can handle \SI{30}{\watt} which \emph{should} be enough power. An electrical engineer would need to weigh in
on the PoE requirements.

It appears that most (if not all) 5G USB modems connect using a USB-C interface, while older USB modems typically
used an USB-A interface. My suspicion is the right thing to do would be to use USB-C on an \textmdef{Ideal Router}
and simply require USB modems that use USB-A to use an adapter.

The \tcab{} is usually metal. This will likely cause problems for USB modem
reception \emph{even beyond} the potential interference caused by nearby devices (and frequently, the nearby house
power main).

When using a USB modem, it may be necessary to use a USB modem that supports an antenna located elsewhere with a
connecting cable.

A long USB cable so the USB modem itself can be elsewhere may also work but USB~3.1 seems to have a specified
\SI{18}{\meter} cable limit with most recommendations to keep it under \SI{10}{\meter}. I have seen antennas for 4G
USB modems that have a connecting cable of \SI{30}{\meter} which gives a lot more flexibility (homes with an attic
could put the antenna in the attic, where reception quality is often far better).




Security alarm systems often have cellular Internet connections with such antennas, hopefully
USB modems with such antennas also exist. At least historically they did, but I do not know if they are still made.


\section{House Building Problematic Trend}

From the 1980s to about the late 2000s, it was pretty standard for home builders to put both a telephone jack
and a coaxial jack in every bedroom, plus the living room.

In fact houses built from the late 1990s to the late 2000s often had Cat5e used for the telephone runs, including
a telephone run to the Living Room as Sattelite Television Service often required a `POTS' land-line to confirm
your geographic location and to order PPV content. Those Cat5e cables can usually be reterminated with RJ45 jacks
so that they can deliver wired Ethernet.

A modern trend though seems to be that home builders now only install one coaxial jack to the living room and
\emph{maybe} one coaxial jack to master bedroom. For any additional low voltage wiring, they charge the buyer
extra, with reports (on Reddit) of some charging as much as \$300 \emph{per additional drop}.

If the FCC is going to allow encrypted \atsc{3.0} broadcasts that \emph{require} the \atsc{3.0} tuner have an
Internet connection to decode, the current trend in home building needs to go back to the way things used to
be.

I can sort-of understand home builders not running network cables, but we need to put pressure on local codes
to require that at minimum, an RG6 coaxial cable be run to \emph{every} bedroom plus the living room. That
would at least allow MoCA to provide stable Internet access to any television plugged in to the coaxial jack.

During an emergency, people need to be able to watch television for information. If watching television requires
a stable Internet connection at the tuner, then it needs to be possible to deliver that stable Internet connection
over the same coaxial cable that delivers the broadcast, and the only way that will become normal is if homes
are built with a coaxial cable network providing jacks at all the logical locations where a television might
be placed.

A coaxial cable run to just the living room and master bedroom is not enough. If there is something like a cold
storm and natural gas is cut off, the living room and master bedroom are too big. The family and pets will need
to gather in a smaller bedroom to conserve heat, and if they still have power, they should be able to watch
local news for instructions.

In 2022, two-hundred and forty-six people died in the
\myhref{https://www.texastribune.org/2022/01/02/texas-winter-storm-final-death-toll-246/}{Texas Winter Storm}.

If a home has power but the furnace is not functioning, watching the news for instructions could be critical
and doing so from within a small room where everyone is gathered could be critical as well.

Dear FCC, for the love of all things good, please do not allow encrypted \atsc{3.0} broadcasts that require
an active Internet connection to decrypt. Internet service itself is frequently lost during emergencies.

\subsection{A Plea to Home Builders---Love Thy Neighbor}

Whether or not local building codes require RG6 coaxial cable drops to every bedroom plus the living room (and I
am hoping people will lobby for that), please commit to doing so.

Furthermore, one more request that I do \emph{not} think should be code but I do think is a mark of a quality
craftsman in the age of `Internet of Things':

At strategic locations within the home, install those round `new work' ceiling junction boxes with a hanger bar
that supports them between joists. Then close to them installed against a joist but above the ceiling, install
one of those standard rectangular 1-gang junction boxes like outlets are often installed in.

Run a Cat6 cable from a logical location for a \textmdef{LAN distribution switch} to the standard rectangular
1-gang junction box, terminated in that junction box with an RJ45 keystone jack. Then run a short Cat6 patch cable
with manufactured ends from the jack in that box into the ceiling junction box, and put a cover plate on it.

That way if the resident ever wants quality WiFi in their house, they can install ceiling-mounted Wireless Access
Points that are powered by PoE onto those junction boxes \emph{without} needing to cut the ceiling holes and running
the network cables themselves.

Cat6 should be used rather than Cat5e as some WiFi hot spots support data transfer rates that exceed
\SI{1}{\giga\bit\per\second}.

WiFi in houses has become standard and many appliances now want it. Quality workmanship will take that into
consideration when building the house and pre-wire the house for ceiling-mounted access points in locations
to offer full-house coverage.

Don't forget the ceiling in any garages.






































\chapter{A New Triple Play}

In the sport of baseball---often referred to as the `American Sport'---a Triple
Play is a very rare event that only happens a handful of times every season. It
is a very exciting event to witness.

Cable Television companies capitalized off of the excitement of a Triple Play to
offer a so-called `Triple Play' bundle: Cable Television, Internet Service, Home
Phone Service. The home phone service used VoIP technology to allow a traditional
`Tip and Ring' analog phone to receive a emulated analog phone service over the
Internet.

With the rapid adoption of cellular phone service, more and more people just are
not interested in traditional `Tip and Ring' analog phone service and for those
who are, third party ATA (Analog Telephone Adapter) hardware is inexpensive and
frequently far superior to the ATA provided by the `Triple Play' bundle.

During the late 1990s when Cable Modem service first became available to residential
customers, it was a godsend. Times have changed, and now for many---though admittedly
not yet all---residential customers, Fiber Optic Internet service is available and
is a much superior technology for Internet service.

With the proliferation of broad-band Internet, streaming services that are much
cheaper than Cable Television service and do not require the rental of a Cable Television
tuner have proliferated resulting in many people `Cutting the Cable'.

To meet the demand of high-speed Internet over a coaxial cable, the DOCSIS standard
has resulted in a reduction of available RF frequencies for television content,
often resulting in an over-compression of television content which results in a
reduced picture quality to the point where OTA antenna service now produces a visually
sharper image compared to Cable Television.

The over-compression of television content is a hack by the cable companies that
degrages the experience of their customers.

To meet the demand of high-speed Internet over a coaxial cable, \docsis{3.1} and
\docsis{4.0} frequencies are in conflict with the RF frequencies used by
\xdband{} and the higher RF frequencies are not compatible with many of the
existing coaxial surge suppressors currently installed at the coaxial cable
demarcation point.

To get around the \xdband{} conflict, a two-way splitter is needed at the demarcation
point to provide a dedicated DOCSIS jack. When an incompatible surge suppressor is
used at the demarcation point, it either must be removed (reducing electrical safety)
or replaced. Many of the replacement coaxial surge suppressors that are compatible
with \docsis{3.1} `Phase 1' are \emph{not} compatible with the even higher RF frequencies
used by \docsis{3.1} `Phase 2' and \docsis{4.0}.

The increased upper RF range demanded by modern DOCSIS standards is a hack that requires
residential homes to either reduce functionality and/or safety or reconfigure their
coaxial network.

In residential markets where Fiber Optic service is available, a rapidly increasing
market, I believe it is time for Cable providers to redefine their strategy and
create a new `Triple Play' to market to consumers:

\begin{enumerate}
  \item Public Broadcast Service over Coax (see page~\pageref{apx:triple:broadcast})
  \item Broadband Internet Service over Fiber Optic (see page~\pageref{apx:triple:internet})
  \item Non-Public Broadcast Television over Broadband Internet
\end{enumerate}

Such a package would greatly benefit the consumer and could reverse the `Cut the Cable'
trend.

Please note that for residential communities where Cable service is available but where
Fiber Optic service is not available---I do feel it is \emph{critical} that at a minimum,
\docsis{3.0} service and preferably at least \docsis{3.1} service remain an option.

\section{Public Broadcast Service over Coax}
\label{apx:triple:broadcast}

After the 2020 repack, broadcast television within the United States uses a lower--upper edge
frequency range of \atscrange{}. Within that frequency range and not in conflict
with any broadcast television frequencies is the FM radio band, using \fmrange{}.

Unless there is another repack to get rid of VHF-Lo (something I would advocate, most consumer
indoor antennas currently on the market can not adequately receive it anyway) the
\atscrange{} should be reserved for Public Broadcast Service.

For Cable subscribers who purchase `Public Broadcast Service over Coax', the cable companies
would be \emph{required} to provide the same signal within the \atscrange{} for the DMA (Designated
Market Area) that is broadcast over the public airwaves \emph{including} FM radio transmissions.

Any subscriber to this service should be able to attach their coaxial cable to a television set (or
any \atsc{1.0}/\atsc{3.0} tuner) and tune all of the public broadcast content as if they were using
an antenna. Any subscriber to this service should be able to attach their coaxial cable to radio
that has a \SI{75}{\ohm} coaxial antenna jack and receive all of the public broadcast content as if
they were using an antenna.

Cable companies could \emph{optionally} offer out-of-market content and non-broadcast content on
unused frequencies within the \atscrange{} or even the historic broadcast television range above
\SI{608}{\mega\hertz}.

For example, to encourage customers to subscribe, they could make a deal with CSPAN to offer
CSPAN on, say, the historic broadcast channel 40 (\SIrange{626}{632}{\mega\hertz} in ATSC 3.0
format.

Television channels in the historic \SIrange{614}{890}{\mega\hertz} range could even become a
source of revenue for cable companies.

If someone has content they think can earn them advertising revenue but they can not get that
content onto a broadcast tower for the DMA, they could buy a channel from the cable company and
send their ATSC signal through the cable company.

I am fundamentally opposed to encrypted broadcasts, but for content that is \emph{not} broadcast
in the DMA yet made available as an ATSC signal, I \emph{personally} would not care if it used
encrypted \atsc{3.0}.

\subsection{Benefits to the Public Interest}

Broadcast Television and Broadcast Radio are \emph{critical} services for the dissemination of
information to the public during a crisis and as a means to exercise our first amendment right to
the freedom of speech.

When I was growing up in the 1980s, KTVU had a public service where they would allow ordinary
regular people to come in and give their perspective on a local political issue.

I absolutely loved it because it made me feel empowered. People like me who were not CEOs, people
like me who were not wealthy, we still had a means by which our voice could be heard.

That's important and no, social media most definitely \emph{does not} serve that same function
today.

Broadcast Television and Broadcast Radio are \emph{critical} services for the dissemination of
information to the public, but they require an antenna.

Outdoor antennas are a potential source of danger during an electrical storm.

Even when NEC code is strictly followed and both the antenna mast and coaxial cables are properly
bonded to an NEC source of ground, there is still a risk of damage caused by a lightning strike
entering the home through the antenna coaxial system.

The same broadcasts delivered through the cable company does require that coaxial feed be grounded
where it enters the home, but the risk of fire during an electrical storm is lower than the risk
of fire from an external antenna \emph{especially} because cable companies do a pretty good job of
making sure the demarcation point is in fact properly bonded to an NEC compliant source of ground.

Broadcast Television and Broadcast Radio are \emph{critical} services for the dissemination of
information to the public, but they require an antenna.

Many people live in apartment buildings on a side of the building where indoor antennas simply
do not work and there also is no suitable location to install an outdoor antenna. Broadcast
television and broadcast radio service delivered over a coaxial cable from the cable company
would allow those people access to the broadcast television and broadcast radio within their
DMA.

Broadcast Television and Broadcast Radio are \emph{critical} services for the dissemination of
information to the public, but they require an antenna.

Sometimes there exist geographical or other barriers that prevent an antenna from receiving
a signal.

For example, I live in East Contra Costa County. My DMA is the `San Francisco-Oakland-San Jose
Media Market Area'. The broadcasters for my DMA are all \SIrange{40}{50}{\mile} away with
Mount Diablo sitting between me and those broadcast towers. Reception might be possible with
a very expensive Yagi-Uda installed on the roof, but the RabbitEars.info report for stations
in my DMA are all listed as `Poor' or `Bad'.

I'm luckier than many people in a similar situation, as there is an antenna cluster in Walnut
Grove within \SIrange{20}{25}{\mile} and no mountains to block the signal, but technically
those broadcast antennas serve the Sacramento-Stockton-Modesto DMA.

Much of the television content I receive is the same, but there are some very important
differences:

\begin{itemize}
  \item I do not receive `Emergency Broadcast System' warnings for my DMA unless the warning
        is also sent to the Sacramento-Stockton-Modesto DMA. This could cause injury or loss
        of life in the event of a wildfire or flood that impacts one DMA but not another.
  \item I do not receive local news. This makes it more difficult to follow local political
        issues and denies local political campaigns the ability to reach me.
  \item I do not receive advertisements from local small businesses, I only receive advertisements
        for small businesses \emph{outside} of my DMA. That hurts local small businesses.
\end{itemize}

There used to be a non-profit streaming service called \myhref{https://en.wikipedia.org/wiki/Locast}{Locast}
that solved those issues at least for people with broadband Internet service, but unfortunately
Locast was found to be in violation of copyright law and was shut down.

Broadcast Television and Broadcast Radio are \emph{critical} services for the dissemination of
information to the public. Cable companies \emph{already} have the necessary infrastructure to
get those broadcasts to people \emph{without} the need for an antenna, but the business model
that cable companies currently have is highly flawed---likely as a result of their decades-long
near monoplolistic hold on their markets---and the net result is that now they are losing
customers and many people are cut off from these \emph{critical} services.

Where broadband Internet is available, Internet streaming services are genuinely a better technology
for the delivery of paid non-broadcast content to the consumer than through Cable company coax.

Where Fiber Optic is available, Fiber Optic service is genuinely a better technology for the delivery
of broadband Internet than DOCSIS through Cable company coax.

Cable companies need to realize their model is now outdated and adapt.

Expensive cable-box rentals required to even receive local programming is probably the largest
driving force in the `cut the cable' movement. Nobody wants to pay an expensive monthly fee to
get content that is better delivered over an Internet connection \emph{without} needing to
pay a rental box fee.

Lack of visual quality for local broadcasts compared to what is available from an antenna is
also a factor.

The Cable companies need to adapt. To quote Bob Dylan:

\begin{quotation}
\noindent Come gather 'round people wherever you roam\\
And admit that the waters around you have grown\\
And accept it that soon you'll be drenched to the bone\\
If your time to you is worth savin'\\
And you better start swimmin' or you'll sink like a stone\\
For the times they are a-changin'
\end{quotation}


\section{Broadband Internet Service over Fiber Optic}
\label{apx:triple:internet}































\end{document}
