\chapter{A New Triple Play}

In the sport of baseball---often referred to as \textit{America's National Pastime}---a
Triple Play is a very rare event that only happens a handful of times every season.
It is a very exciting event to witness.

Cable Television companies capitalized off of the excitement of a Triple Play to
offer a so-called `Triple Play' bundle: Cable Television, Internet Service, Home
Phone Service. The home phone service used VoIP technology to allow a traditional
`Tip and Ring' analog phone to receive a emulated analog phone service over the
Internet.

With the rapid adoption of cellular phone service, more and more people just are
not interested in traditional `Tip and Ring' analog phone service and for those
who are, third party ATA (Analog Telephone Adapter) hardware is inexpensive and
frequently far superior to the ATA provided by the `Triple Play' bundle.

During the late 1990s when Cable Modem service first became available to residential
customers, it was a godsend. Times have changed, and now for many---though admittedly
not yet all---residential customers, Fiber Optic Internet service is available and
is a much superior technology for Internet service.

With the proliferation of broad-band Internet, streaming services that are much
cheaper than Cable Television service and do not require the rental of a Cable Television
tuner have proliferated resulting in many people `Cutting the Cable'.

To meet the demand of high-speed Internet over a coaxial cable, the DOCSIS standard
has resulted in a reduction of available RF frequencies for television content,
often resulting in an over-compression of television content which results in a
reduced picture quality to the point where OTA antenna service now produces a visually
sharper image compared to Cable Television.

The over-compression of television content is a hack by the cable companies that
degrages the experience of their customers.

To meet the demand of high-speed Internet over a coaxial cable, \docsis{3.1} and
\docsis{4.0} frequencies are in conflict with the RF frequencies used by
\xdband{} and the higher RF frequencies are not compatible with many of the
existing coaxial surge suppressors currently installed at the coaxial cable
demarcation point.

To get around the \xdband{} conflict, a two-way splitter is needed at the demarcation
point to provide a dedicated DOCSIS jack. When an incompatible surge suppressor is
used at the demarcation point, it either must be removed (reducing electrical safety)
or replaced. Many of the replacement coaxial surge suppressors that are compatible
with \docsis{3.1} `Phase 1' are \emph{not} compatible with the even higher RF frequencies
used by \docsis{3.1} `Phase 2' and \docsis{4.0}.

The increased upper RF range demanded by modern DOCSIS standards is a hack that requires
residential homes to either reduce functionality and/or safety or reconfigure their
coaxial network.

In residential markets where Fiber Optic service is available, a rapidly increasing
market, I believe it is time for Cable providers to redefine their strategy and
create a new `Triple Play' to market to consumers:

\begin{enumerate}
  \item Public Broadcast Service over Coax (see page~\pageref{apx:triple:broadcast})
  \item Broadband Internet Service over Fiber Optic (see page~\pageref{apx:triple:internet})
  \item Non-Public Broadcast Television over Broadband Internet
\end{enumerate}

Such a package would greatly benefit the consumer and could reverse the `Cut the Cable'
trend.

Please note that for residential communities where Cable service is available but where
Fiber Optic service is not available---I do feel it is \emph{critical} that at a minimum,
\docsis{3.0} service and preferably at least \docsis{3.1} service remain an option.

\section{Public Broadcast Service over Coax}
\label{apx:triple:broadcast}

After the 2020 repack, broadcast television within the United States uses a lower--upper edge
frequency range of \atscrange{}. Within that frequency range and not in conflict
with any broadcast television frequencies is the FM radio band, using \fmrange{}.

Unless there is another repack to get rid of VHF-Lo (something I would advocate, most consumer
indoor antennas currently on the market can not adequately receive it anyway) the
\atscrange{} should be reserved for Public Broadcast Service.

For Cable subscribers who purchase `Public Broadcast Service over Coax', the cable companies
would be \emph{required} to provide the same signal within the \atscrange{} for the DMA (Designated
Market Area) that is broadcast over the public airwaves \emph{including} FM radio transmissions.

Any subscriber to this service should be able to attach their coaxial cable to a television set (or
any \atsc{1.0}/\atsc{3.0} tuner) and tune all of the public broadcast content as if they were using
an antenna. Any subscriber to this service should be able to attach their coaxial cable to radio
that has a \qty{75}{\ohm} coaxial antenna jack and receive all of the public broadcast content as if
they were using an antenna.

Cable companies could \emph{optionally} offer out-of-market content and non-broadcast content on
unused frequencies within the \atscrange{} or even the historic broadcast television range above
\qty{608}{\mega\hertz}.

For example, to encourage customers to subscribe, they could make a deal with CSPAN to offer
CSPAN on, say, the historic broadcast channel 40 (\qtyrange{626}{632}{\mega\hertz} in ATSC 3.0
format.

Television channels in the historic \qtyrange{614}{890}{\mega\hertz} range could even become a
source of revenue for cable companies.

If someone has content they think can earn them advertising revenue but they can not get that
content onto a broadcast tower for the DMA, they could buy a channel from the cable company and
send their ATSC signal through the cable company.

I am fundamentally opposed to encrypted broadcasts, but for content that is \emph{not} broadcast
in the DMA yet made available as an ATSC signal, I \emph{personally} would not care if it used
encrypted \atsc{3.0}.

\subsection{Benefits to the Public Interest}

Broadcast Television and Broadcast Radio are \emph{critical} services for the dissemination of
information to the public during a crisis and as a means to exercise our first amendment right to
the freedom of speech.

When I was growing up in the 1980s, KTVU had a public service where they would allow ordinary
regular people to come in and give their perspective on a local political issue.

I absolutely loved it because it made me feel empowered. People like me who were not CEOs, people
like me who were not wealthy, we still had a means by which our voice could be heard.

That's important and no, social media most definitely \emph{does not} serve that same function
today.

Broadcast Television and Broadcast Radio are \emph{critical} services for the dissemination of
information to the public, but they require an antenna.

Outdoor antennas are a potential source of danger during an electrical storm.

Even when NEC code is strictly followed and both the antenna mast and coaxial cables are properly
bonded to an NEC source of ground, there is still a risk of damage caused by a lightning strike
entering the home through the antenna coaxial system.

The same broadcasts delivered through the cable company does require that coaxial feed be grounded
where it enters the home, but the risk of fire during an electrical storm is lower than the risk
of fire from an external antenna \emph{especially} because cable companies do a pretty good job of
making sure the demarcation point is in fact properly bonded to an NEC compliant source of ground.

Broadcast Television and Broadcast Radio are \emph{critical} services for the dissemination of
information to the public, but they require an antenna.

Many people live in apartment buildings on a side of the building where indoor antennas simply
do not work and there also is no suitable location to install an outdoor antenna. Broadcast
television and broadcast radio service delivered over a coaxial cable from the cable company
would allow those people access to the broadcast television and broadcast radio within their
DMA.

Broadcast Television and Broadcast Radio are \emph{critical} services for the dissemination of
information to the public, but they require an antenna.

Sometimes there exist geographical or other barriers that prevent an antenna from receiving
a signal.

For example, I live in East Contra Costa County. My DMA is the `San Francisco-Oakland-San Jose
Media Market Area'. The broadcasters for my DMA are all \qtyrange{40}{50}{\mile} away with
Mount Diablo sitting between me and those broadcast towers. Reception might be possible with
a very expensive Yagi-Uda installed on the roof, but the RabbitEars.info report for stations
in my DMA are all listed as `Poor' or `Bad'.

I'm luckier than many people in a similar situation, as there is an antenna cluster in Walnut
Grove within \qtyrange{20}{25}{\mile} and no mountains to block the signal, but technically
those broadcast antennas serve the Sacramento-Stockton-Modesto DMA.

Much of the television content I receive is the same, but there are some very important
differences:

\begin{itemize}
  \item I do not receive `Emergency Broadcast System' warnings for my DMA unless the warning
        is also sent to the Sacramento-Stockton-Modesto DMA. This could cause injury or loss
        of life in the event of a wildfire or flood that impacts one DMA but not another.
  \item I do not receive local news. This makes it more difficult to follow local political
        issues and denies local political campaigns the ability to reach me.
  \item I do not receive advertisements from local small businesses, I only receive advertisements
        for small businesses \emph{outside} of my DMA. That hurts local small businesses.
\end{itemize}

There used to be a non-profit streaming service called \myhref{https://en.wikipedia.org/wiki/Locast}{Locast}
that solved those issues at least for people with broadband Internet service, but unfortunately
Locast was found to be in violation of copyright law and was shut down.

Broadcast Television and Broadcast Radio are \emph{critical} services for the dissemination of
information to the public. Cable companies \emph{already} have the necessary infrastructure to
get those broadcasts to people \emph{without} the need for an antenna, but the business model
that cable companies currently have is highly flawed---likely as a result of their decades-long
near monoplolistic hold on their markets---and the net result is that now they are losing
customers and many people are cut off from these \emph{critical} services.

Where broadband Internet is available, Internet streaming services are genuinely a better technology
for the delivery of paid non-broadcast content to the consumer than through Cable company coax.

Where Fiber Optic is available, Fiber Optic service is genuinely a better technology for the delivery
of broadband Internet than DOCSIS through Cable company coax.

Cable companies need to realize their model is now outdated and adapt.

Expensive cable-box rentals required to even receive local programming is probably the largest
driving force in the `cut the cable' movement. Nobody wants to pay an expensive monthly fee to
get content that is better delivered over an Internet connection \emph{without} needing to
pay a rental box fee.

Lack of visual quality for local broadcasts compared to what is available from an antenna is
also a factor.

The Cable companies need to adapt. To quote Bob Dylan:

\begin{quotation}
\noindent Come gather 'round people wherever you roam\\
And admit that the waters around you have grown\\
And accept it that soon you'll be drenched to the bone\\
If your time to you is worth savin'\\
And you better start swimmin' or you'll sink like a stone\\
For the times they are a-changin'
\end{quotation}


\section{Broadband Internet Service over Fiber Optic}
\label{apx:triple:internet}























