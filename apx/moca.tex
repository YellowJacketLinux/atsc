\chapter{A MoCA Compromise}
\label{apx:moca}

If the FCC decides to allow broadcasters to continue to broadcast encrypted \atsc{3.0}
then it is \textbf{imperative} that the FCC require
\ngtv{} certified tuners to implement \xdband{}~2.5 as a means by which to
assist households in achieving a stable Internet service at the \atsc{3.0} tuner.

It is my \emph{hope} that those behind \ngtv{} certification will see the wisdom in
implementing MoCA as part of the standard even without pressure from the FCC.

DirecTV `SWM' set-top boxes have built-in \eband{} clients (sometimes referred to as DECA),
Dish Network `Hopper and Joey' set-top boxes have built-in \fband{} clients, and many Cable
Television and TiVO set-top boxes have either built-in \dband{} or \xdband{} clients.

MoCA is an excellent networking solution when coaxial cables are involved. The Commercial
television delivery services that depend upon an Internet connection know this and have
implemented it, but I have not been able to find a single \atsc{1.0} or \atsc{3.0} tuner
that implements MoCA. That needs to change.

Many housing units in the United States of America already have coaxial cable infrastructure
installed and this infrastructure usually originates in a location close to the same
NEC compliant source of ground that an outdoor antenna is required by code to be bonded
to. This allows outdoor (or attic) antennas to connect to the distribution coaxial splitter
instead of Cable Television service, and also make it easy to run a MoCA bridge on that coaxial
network.

If all \ngtv{} tuners are \xdband{}~2.5 clients then providing them with stable
Internet access so they can retrieve decryption keys becomes as simple as connecting a
single \xdband{}~2.5 adapter to both the coaxial distribution splitter and to the
Internet router.

\ngtv{} tuners should use \moca{2.5} rather than \moca{2.0} so that users who want additional
security can enable \textmdef{MoCA Protected Setup} (aka MPS). \moca{3.0} can optionally
be implemented but would be wasteful on \atsc{3.0} tuners that have no need for the
bandwidth capabilities beyond \moca{2.5}. \xdband{} should be used because it does not collide
with the RF frequencies used by broadcast television.

As \moca{2.5} networks are limited to 16 nodes, \atsc{3.0} tuners should have the ability to
disable their built-in MoCA client \emph{even when a MoCA network is present} just in case
the household is using MoCA for lots of other things too and is running low on nodes.

\ngtv{} tuners should continue to offer WiFi connectivity on the \qty{2.4}{\giga\hertz} band
and the \qty{5}{\giga\hertz} band and ideally should also offer an Ethernet jack so that
Internet access can still be achieved in households where MoCA is not a workable solution---such
as households that use one indoor antenna per television and do not use a coaxial cable network.

If \ngtv{} requires MoCA clients be coupled with \atsc{3.0} tuners for certification, then the
odds of a tuner being capable of retrieving the decryption keys even when the tuner located in
a WiFi dead spot or problem spot within the home will increase significantly.

If someone can't change the channel because the microwave is on, it will be very annoying. That
sometimes happens with \qty{2.4}{\giga\hertz} WiFi but would not be an issue with MoCA.

If \ngtv{} is going to use an active Internet connection, then \xdband{} \textbf{MUST} be part
of the \ngtv{} standard.


\section{MoCA Retrofit Kit}

Some televisions with \atsc{3.0} tuners and some stand-alone \atsc{3.0} tuners that bear
a \ngtv{} certification are already on the market, and I suspect many more have
already been designed for the 2024 model year and it would be a financial burden to
forbid those products from hitting the market.

For such devices, I recommend a subsidized \textmdef{MoCA Retrofit Kit} be provided to
low-income households who have such devices. Unfortunately, this kit would be
useless for \atsc{3.0} tuners that lack an Ethernet jack.

Specific products mentioned below are simply products that I have personal experience with.
It is possible there are products from other manufacturers that are better suited for this
purpose.

The kit should include the following components:

\begin{description}
  \item[One Passive Two-Way MoCA Splitter] \hfill \\ I have personal experience with the
    \myhref{https://www.antronix.com/pdf/ds-1181-ss-a02-mmc1000-splitters.pdf}{Antronix MMC1002H-B}
    splitter to provide D-Band MoCA on `OTA Broadcast Television' coaxial networks. It works
    well.
  \item[One \xdband{}~2.5 Adapter] \hfill \\ I have personal experience with the
    \myhref{https://www.screenbeam.com/products/ecb6250/}{ScreenBeam ECB6250}
    MoCA 2.5 Network Adapters. That particular model has a \qty{1}{\giga\bit\per\second} Ethernet jack.
    They also make a more expensive model with a \qty{2.5}{\giga\bit\per\second} Ethernet jack
    (the ECB7250) but no \atsc{2.5} tuner needs that capability.
  \item[Two 18 Inch RG6 Coaxial Cables] \hfill \\
    One is needed to connect the television to the passive two-way splitter, one is needed to
    connect the MoCA Adapter to the passive two-way splitter.
  \item[One 3 foot Cat5e Patch Cable] \hfill \\
    Cat5e patch cables are far more flexible than Cat6 patch cables and are perfectly capable of
    handling the network speeds required of a MoCA 2.5 network.
\end{description}

In a nutshell, rather than having the antenna cable feeding the \atsc{3.0} tuner directly, it would
feed the two-way passive splitter. One split would then feed the \atsc{3.0} tuner, and the other
split would feed the \xdband{}~2.5 adapter which in turn would provide an Ethernet connection
to the \atsc{3.0} tuner so that the tuner, thus allowing decryption of encrypted \atsc{3.0}
broadcasts in cases where the tuner can not get a reliable WiFi connection.

\subsection{Cost Estimate}

Pricing those items (excluding the cables) at Amazon comes to about \$90.

That cost could probably be significantly reduced if purchased in bulk from the product manufacturers
but I do not \emph{personally} have the business experience needed to estimate such a cost reduction
and I would prefer not to speculate in a document I am sending to the government.

The ECB6250 as sold at Amazon comes with one coaxial cable and one Ethernet cable. The coaxial
cable that it comes with---to be frank---is junk. The cable is only dual-shield RG6 and thus susceptible
to RF interference from power cables which is likely when installed in the proximity of a television,
and the connectors on it are junk.

A 250 foot pull box of Mediabridge Quad-Shield RG6 costs about \$66 and would produce at least 150
\qty{18}{inch} cables, enough for 75 \textmdef{Moca Retrofit Kits}. That is 88¢ per kit.

A box of 50 quality Klein compression F connectors for quad-shield coax costs about \$45 and would
provide for 12 \textmdef{Moca Retrofit Kits}. That is \$3.75 per kit.

Ignoring labor, manually created quality RG6 cables would thus cost under \$5 per kit.

Again, I simply do not have the business experience to estimate how much that cost would be reduced
by ordering from a cable manufacturer that produces them in bulk by machine, but I would be very
surprised if the cost was not cut down by more than half. Quality quad-shield RG6 cables purchased in bulk
would not significantly add to the cost of the \textmdef{Moca Retrofit Kit} and would greatly decrease
interference problems that need to be solved post-install.

The Ethernet cable that the ECB6250 kit comes with is probably sufficient but quality patch cables
purchased in bulk are extremely cheap, they are heavily used in today's high-tech industry and are
mass-produced by numerous companies.

Three foot Cat5e patch cables purchased in quantities of 500 cables from
\myhref{https://rb.gy/s18bc}{CableWholesale} currently cost \$1.13 each and they do offer further
discounts for larger quantities.

\subsection{Fire Safety}

During my early twenties, I lived for a time near Home, Washington (Key Peninsula). My back caught
on fire. The fire was the result of inadequate safety that was a direct result of the poverty conditions
I was living under at the time.

In my case, the fire was the result of wearing a flanel shirt that had likely absorbed gasoline fumes from
power tools we were using to build a shelter, then ignited by me standing too close to a heat source on a
cold night when we did not yet have adequate shelter. I should have known better, but poverty sucks.
Fortunately one of the people living there was a former EMT who had lost his job and was homeless as a
result, so I did get the best possible first aid when it happened---and the EMTs in the ambulance and the
doctors were also top notch.

Ever since that incident though I have been paranoid about household fires, which seem to disproportionally
impact those living in poverty. As the proposed \textmdef{MoCA Retrofit Kit} is intended for those living in
poverty, fire safety should in my opinion be evaluated.

There are two fire safety concerns I have with the AC to DC transformer that comes with the previously
mentioned `ScreenBeam ECB6250' kit. Whether these concerns are genuine or not, I can not say, I may
be being overly paranoid.

The first concern I have is that I see no indication on the transformer itself that it has overcurrent
protection. It is true that Class~\rom{2} transformers are not capable of producing enough current to
electrocute someone but if there is a short in the device being powered, fire is still possible.

My second concern is that the transformer included with the `ScreenBeam ECB6250' kit is designed to fit
standard two-outlet wall receptable such that the body of the transformer does not block the use of
the outlet receptacle it is not plugged into.

When used for a television, that may actually be a problem. Most television areas have multiple items
that need to be plugged in, resulting in the use of a power bar. With most power bars, the outlet
receptacles are arranged such that the transformer body would block adjacent receptacles. This would
tempt the person to use a short extension cord between the transformer and the power bar, increasing
the odds of a loose connection causing an arc-fault.

The NEC has required arc-fault circuit breakers for most circuits in a home since 1999 but NEC is
not law and local code often lags behind the NEC. Furthermore, until 2007 or so (I might not be
correctly remembering) most arc-fault circuit breakers did not do a very good job at detecting
arc-faults in extensions to a circuit. Finally, those living in poverty are less likely to be
living in housing built after arc-fault protected circuits became code.

What I personally did in my current home with MoCA adapters, I used a
\myamazon{B00QV5Q5HU}{Ugreen} AC to DC transformer that has identical
electrical specifications (\qty{5}{\volt}, \qty{2}{\ampere}) with an identical positive-center
DC barrel (OD \qty{3.55}{\milli\meter}, ID \qty{1.35}{\milli\meter}) that has overcurrent protection
and has the transformer positioned such that it does not block adjacent outlet sockets on a
typical power bar.

Ideally, housing without arc-fault circuit breakers should install
\myamazon{B01CG8MP9W}{AFCI receptacles}
wherever power bars are used---cheaply manufactured power bars themselves are a common source of
arc-fault fires in addition to issues with cords not being properly plugged in---but that is beyond
the scope of the FCC or the Broadcast Television Industry.

\subsection{Signal Strength}
\label{apx:moca:signalstrength}

An unfortunate consequence of a passive two-way splitter is a reduction in the broadcast signal
strength. The actual signal attenuation that results varies by frequency but it will \emph{always}
be at least \qty{3}{\decibel} and the number Antronix prints on their two-way MoCA splitter is
\qty{3.5}{\decibel}.

When a \qty{3.5}{\decibel} loss in broadcast signal strength is an issue, it \emph{usually} can
be resolved by using a slightly more powerful amplifier, installing a better antenna, or a
combination of the two.

A technically better solution would be to use a band separator rather than a passive two-way
splitter.

As far as physical connections are concerned, a band separator is virtually identical to a
passive two-way splitter. The difference is with a band separator, RF frequencies below a
specific target are separated to one output and RF frequencies above that target are separated
to the other target.

An example of a band separator for coaxial cables is the
\myhref{https://pixelsatradio.com/products/am-fm-band-separator}{Pixel Technologies AMFMBS}.
That band separator is for radio broadcast bands and is not suitable for this application.

The reason why my recommended \textmdef{MoCA Retrofit Kit} specifies a passive two-way splitter
instead of a band separator is simply because I am not aware of an existing external coaxial
band separator that separates the ATSC Broadcast Band from \xdband{} and I suspect there
is not enough market demand to produce such a product at an affordable price. One certainly
could be produced, I just question if demand would warrant a supply production at an affordable
per-unit price.





