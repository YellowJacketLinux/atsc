\chapter*{abstract}

From the \myhref{https://www.fcc.gov/media/radio/public-and-broadcasting}{FCC Website}:

\begin{quotation}
\noindent
In exchange for obtaining a valuable license to operate a broadcast station using the public airwaves,
each radio and television licensee is required by law to operate its station in the ``public interest,
convenience and necessity.'' \hfill \\
\ldots \hfill \\
Station licensees, as the trustees of the public’s airwaves, must use the broadcast medium to serve the
public interest.  We at the FCC want you to become involved\ldots
\end{quotation}

During the first decade of this century, broadcast television in the United States transitioned from
analog broadcasts to digital broadcasts. This transition was in the best interest of the public, both
better picture quality and an increase of available content was made available as a result of the
superior compression available by using an encoded digital signal rather than an analog signal.

The transition did have a drawback: All consumers of broadcast television either needed to purchase
a new television capable of tuning \atsc{1.0} broadcasts, or alternatively purchase an external converter
box that hooked up to their existing television through an HDMI cable or RCA cables for cables that did
not have an HDMI interface, and some households in rural areas were no longer able to receive broadcast
television.

Some people were unable to afford the converter boxes. A subsidy program existed that helped these
people with a \$40 subsidy towards the converter box.

According to Nielsen Media Research. 3.1\% of Americans were not yet equipped for the transition to
digital broadcasts by the mandated cut-off date. I was actually one of those people, but I did not
sweat it because I was planning on purchasing a new television as soon as I could. I ended up needing
to wait until 2015 to get a new television, and thus went five years without television of any kind.

Now here we are just under a decade and a half since the last transition finished and we are
undergoing a new transition, the transition from \atsc{1.0} to \atsc{3.0}.

Like the transition from analog to \atsc{1.0}, there are definite technological benefits that serve
the public interest. However the transition from \atsc{1.0} to \atsc{3.0} is not just as simple as
buying a new television or a new converter box, many \atsc{3.0} broadcasts are encrypted and the
decryption of the signal \emph{requires} an active Internet connection that the tuner can use to
fetch the decryption key.

Particularly within the economically depressed population, there are many people who do not have
Internet service at their home. For those who do have Internet service at their home, they can
not always afford to pay the bill on time and sometimes they lose service. Even for those who
do have the financial means to pay their Internet bill on time, they may not have reliable service
that their television is capable of using. And finally, during disasters, it is not too uncommon
for Internet service to go out.

I do not believe it is in the `public interest' for the FCC to allow encrypted broadcasts on the
public airways. If the broadcasters are willing to continue broadcasting \atsc{1.0} along with
the encrypted \atsc{3.0} then \emph{maybe} it might be okay, but that is not what public broadcasters
wish to do. They want to terminate \atsc{1.0} broadcasts.

% stuff here

Appendix~\ref{apx:moca} (page~\pageref{apx:moca}) describes a possible update to the \ngtv{}
specification that would make it easier to get reliable Internet service to an \atsc{3.0}
tumer.

Appendix~\ref{apx:router} (page~\pageref{apx:router}) probably should be a separate document,
but it addresses changes in the Home Internet market that really need to be made to increase
the chances of the compromise suggested in Appendix~\ref{apx:moca} actually working.

\section*{PDF Version}
PDF versions of this document are available from \textit{update needed}. Two versions of the
PDF are made available: One version is compiled for double-sided printing on US Letter and the
other version is compiled for single-sided printing on US Letter. It is possible to modify
the source for printing on other paper sizes.

For members of the general public who read this, I ask that you print copies to send to your
government and to the FCC, preferably with your own cover letter. If they get a lot of copies
of this document sent to them from lots of different people, maybe just maybe they will actually
pay attention.

The PDF files are monochrome. Color printing is not required.


\section*{Hyperlink Note}

For typographical reasons, some hyperlinks had to be shortened using a URL shortener. Most URL
shorteners do in fact use tracking cookies. The domain used for the URL shortener is
\texttt{rb.gy}.

For privacy, you may wish to delete cookies in your browser that belong to that domain if you
follow any of those shortened hyperlinks.


\section*{Document Source}

The \LaTeX{} source for this document can be found at on
\myhref{https://github.com/YellowJacketLinux/atsc}{github}.
